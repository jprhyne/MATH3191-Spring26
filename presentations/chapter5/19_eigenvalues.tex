\documentclass[xcoler=dvipsnames, aspectratio=169]{beamer}

\usepackage{3191Style}
% Date gives the title of the lecture
\date{Eigenvalues and Eigenvectors}

\begin{document}
    \begin{frame}{Eigenvalues and Eigenvectors\footnote{Aside: The word ``eigen'' comes from 
        German and roughly translates to either ``own/self'' or ``characteristic''.}}
        \begin{defn}
            Let $A\in\R^{n\times n}$, then we define:
            \begin{enumerate}
                \pause\item An \rText{eigenvector} of $A$ is a vector $\vec{v}\neq\vec{0}$ such that
                    $A\vec{v} = \lambda\vec{v}$ for some scalar $\lambda$.
                \pause\item An \rText{eigenvalue} of $A$ is a scalar $\lambda$ such that there is
                    some $\vec{v}\neq\vec{0}$ such that $A\vec{v} = \lambda\vec{v}$.
            \end{enumerate}
        \end{defn}\pause
        \begin{tcolorbox}
            Note: Since the definitions of eigenvalues and eigenvectors depend on each other, we sometimes
            refer to the ordered pair
            \[
                (\lambda,\vec{v})
            \]
            as an \bText{eigenpair} of $A$.
        \end{tcolorbox}
    \end{frame}
    \begin{frame}{Another Framing of Eigenvalues and Eigenvectors}
        From the definition of the eigenpair $(\lambda,\vec{v})$, we see that $\vec{v}$ is a non-trivial
        solution to the system
        \[
            \vec{0} = A\vec{v} - \lambda\vec{v} = (A-\lambda I_n)\vec{v}
        \]\pause
        So, if $\lambda$ is an eigenvalue of $A$, then the matrix $A-\lambda I$ has a non-trivial
        null space, and the eigenvectors will be the vectors of this null space!
    \end{frame}
    \begin{frame}{Finding Eigenvectors for an Eigenvalue Example}
        Let's verify if $\lambda=2$ is an eigenvalue of the following matrix, and if it is, 
        find an eigenvector.
        \[
            A = \bMat{
                3 & 4 & 6\\
                2 & 12&14\\
                1 & 6 &10
            }
        \]\pause
        \[
            \bMat{
                3 & 4 & 6\\
                2 & 12&14\\
                1 & 6 &10
            }\pause\xrightarrow{A=A-\lambda I}\bMat{
                1 & 4 & 6\\
                2 & 10&14\\
                1 & 6 &8
            }\pause\xrightarrow[R_3=R_3-R_1]{R_2=R_2-2R_1}\bMat{
                1 & 4 & 6\\
                0 & 2 & 2\\
                0 & 2 & 2
            }\pause\xrightarrow{R_3=R_3-R_2}\bMat{
                1 & 4 & 6\\
                0 & 2 & 2\\
                0 & 0 & 0
            }
        \]
        So $\lambda=2$ is an eigenvalue! Let's find an eigenvector.
        \[
            \pause\xrightarrow{R_2=\frac{R_2}{2}}\bMat{
                1 & 4 & 6\\
                0 & 1 & 1\\
                0 & 0 & 0
            }\pause\xrightarrow{R_1=R_1-4R_2}\bMat{
                1 & 0 & 2\\
                0 & 1 & 1\\
                0 & 0 & 0
            }\pause\rightarrow \vec{v} = \bMat{-2\\-1\\1}\textnormal{ is an eigenvector of }A.
        \]

    \end{frame}
    \begin{frame}{Finding Eigenvectors for an Eigenvalue Practice}
        Determine if $\lambda=1$ is an eigenvalue of the following matrix and if so, determine
        an eigenvector associated with $\lambda=1$.
        \[
            A = \bMat{
                2 & 2 & 9\\
                2 & 8 &30\\
                1 & 4 &18
            }
        \]
        \iftoggle{showSolutions}{
            \pause
            \[
                \left(1,\bMat{-1\\-4\\1}\right)
            \]
            is an eigenpair. See that\pause
            \[
                A\vec{v} = \bMat{
                    2 & 2 & 9\\
                    2 & 8 &30\\
                    1 & 4 &18
                }\bMat{-1\\-4\\1}\pause = -\bMat{2\\2\\1}-4\bMat{2\\8\\4}+\bMat{9\\30\\18}
                \pause = \bMat{-2-8+9\\-2-32+30\\-1-16+18} = \bMat{
                    -1\\
                    -4\\
                    1
                }\pause = 1\vec{v}
            \]
        }{
            \vspace{130pt}
        }
    \end{frame}
    \begin{frame}{Eigenspace}
        \begin{defn}
            We define the \rText{Eigenspace} of $A$ associated with eigenvalue $\lambda$ to be
            \[
                E\left(A,\lambda\right) = \set{\vec{v}\in\R^n}{A\vec{v} = \lambda\vec{v}}
            \]\pause
            Or equivalently
            \[
                E\left(A,\lambda\right) = \nullS{A-\lambda I_n}
            \]\pause
        \end{defn}
        Since the eigenspace is really just a nullspace, we know how to find a basis of it!
    \end{frame}
    \begin{frame}{Basis for an Eigenspace}
        A basis for an eigenspace of $E(A,\lambda)$ is just a basis for $\nullS{A-\lambda I_n}$.
        So, in order to find such a basis, we can
        \begin{enumerate}
            \pause\item Set up $A-\lambda I_n$.
            \pause\item Reduce to RREF
            \pause\item Write out a basis of this space as before
        \end{enumerate}
    \end{frame}
    \begin{frame}{Basis for an Eigenspace Example}
        Find a basis for the eigenspaces $E(A,1)$ for 
        \[
            A = \bMat{
                7 & 0 & 6\\
                -3& 4 &-6\\
                -3& 0 &-2
            }
        \]\pause
        \[
            \bMat{
                7 & 0 & 6\\
                -3& 4 &-6\\
                -3& 0 &-2
            }\pause\xrightarrow{A=A-\lambda I}\bMat{
                6 & 0 & 6\\
                -3& 3 &-6\\
                -3& 0 &-3
            }\pause\xrightarrow{R_1 = \frac{R_1}{6}}\bMat{
                1 & 0 & 1\\
                -3& 3 &-6\\
                -3& 0 &-3
            }\pause\xrightarrow[R_3=R_3+3R_1]{R_2=R_2+3R_1}\bMat{
                1 & 0 & 1\\
                0 & 3 &-3\\
                0 & 0 & 0
            }
        \]
        \[
            \pause\xrightarrow{R_2=\frac{R_2}{3}}\bMat{
                1 & 0 & 1\\
                0 & 1 &-1\\
                0 & 0 & 0
            }\pause\rightarrow \nullS{A-I} = \Span{\bMat{-1\\1\\1}}
        \]
    \end{frame}
    \begin{frame}{Basis for an Eigenspace Practice}
        Find a basis for the eigenspaces $E(A,4)$ for 
        \[
            A = \bMat{
                7 & 0 & 6\\
                -3& 4 &-6\\
                -3& 0 &-2
            }
        \]
        \iftoggle{showSolutions}{
            \pause
            \[
                E(A,4) = \Span{\bMat{-2\\0\\1}, \bMat{0\\1\\0}}
            \]
        }{\vspace{130pt}}
    \end{frame}
    \begin{frame}{The Eigenspace Associated With $0$}
        What does $E(A,0)$ look like?\pause
        \[
            E(A,0) = \nullS{A-0I} = \nullS{A}
        \]\pause
        So, if this space has a non-trivial basis, then $A\vec{x}=\vec{0}$ has a non-trivial solution!\pause

        Meaning we can say that a matrix is \rText{invertible} if and only if $0$ is \bText{not} an
        eigenvalue of $A$.
    \end{frame}
    \begin{frame}{Linearly Independent Eigenvectors}
        \small
        \begin{theorem}
            If $(\lambda_1, \vec{v}_1)$ and $(\lambda_2,\vec{v}_2)$ are two eigenpairs of a matrix
            $A$ such that $\lambda_1\neq\lambda_2$, then $\vec{v}_1$ and $\vec{v}_2$ are linearly
            independent.
        \end{theorem}\pause
        \begin{proof}
            We will prove this via a contradiction. IE assume $\lambda_1\neq\lambda_2$ but 
            $\vec{v}_1$ and $\vec{v}_2$ are linearly dependent.\pause\ Then, we show that this 
            leads to nonsense.\pause\ If $\vec{v}_1$ and $\vec{v}_2$ are linearly dependent, 
            then there is some constant $c$ such that $\vec{v}_1=c\vec{v}_2$.\pause\ See that
            \[
                \lambda_1\vec{v}_1 = A\vec{v}_1\pause\ = cA\vec{v}_2\pause\ 
                = \lambda_2 c\vec{v}_2 =\pause\ \lambda_2\vec{v}_1
            \]
            So, 
            \[
                \lambda_1\vec{v}_1-\lambda_2\vec{v}_1 = \vec{0}\pause
            \]
            meaning that $\lambda_1=\lambda_2$, which contradicts our assumption that these eigenvalues
            are distinct.
        \end{proof}
    \end{frame}
\end{document}
