\documentclass[xcoler=dvipsnames, aspectratio=169]{beamer}
\usepackage{3191Style}
\date{Applications to Computer Graphics}


\begin{document}

\begin{frame}{Applications of Linear Algebra to Computer Graphics}


\begin{columns}[T]

\column{0.5\tw}

\includegraphics[width=0.8\tw]{../Images/Chap2/fig-horse-triangular.png}

{\tiny \href{https://www.cg.tu-berlin.de/research/projects/harmonic-triangulations/}{\alert{https://www.cg.tu-berlin.de/research/projects/harmonic-triangulations/}}}

\column{0.5\tw}

\includegraphics[width=0.9\tw]{../Images/Chap2/fig-yoda.jpg}

{\tiny Image Credit: Star Wars low poly portraits, designed by Vladan Filipovic}

\bs

\colorb{Let's examine some ways to manipulate and display graphical images using matrices.}

\end{columns}

\end{frame}

\begin{frame}{Scaling One Triangle}

Consider one triangular piece. If we want to scale the triangle by a factor of three, then we apply the transformation
\[ S: \mathbb{R}^2 \to \mathbb{R}^2: \mathbf{x} \mapsto \begin{bmatrix} 3 & 0 \\ 0 & 3 \end{bmatrix} \mathbf{x} .\]
    \pause

\begin{columns}[T]

\column{0.5\tw}

\includegraphics[width=0.9\tw]{../Images/Chap2/fig-triangle-start.png}

\column{0.5\tw}

We have:
\[ \begin{bmatrix} 3 & 0 \\ 0 & 3 \end{bmatrix} \begin{bmatrix} 1 \\ 1 \end{bmatrix} = \begin{bmatrix} 3 \\ 3 \end{bmatrix}\] 

    \pause
\[ \begin{bmatrix} 3 & 0 \\ 0 & 3 \end{bmatrix} \begin{bmatrix} 3 \\ 1 \end{bmatrix} =  \begin{bmatrix} 9 \\ 3 \end{bmatrix} \] 

    \pause

\[ \begin{bmatrix} 3 & 0 \\ 0 & 3 \end{bmatrix} \begin{bmatrix} 2 \\ 3 \end{bmatrix} =  \begin{bmatrix} 6 \\ 9 \end{bmatrix}\]

    \pause
\end{columns}

\end{frame}

\begin{frame}{Vertical Shearing}

If we would like to vertically shear the triangle  by a factor of 2, we have

\[ V: \mathbb{R}^2 \to \mathbb{R}^2: \mathbf{x} \mapsto \begin{bmatrix} 1 & 0 \\ 0 & 2 \end{bmatrix} \mathbf{x} .\]


\begin{columns}[T]

\column{0.5\tw}

We have:
\[ \begin{bmatrix} 1 & 0 \\ 0 & 2 \end{bmatrix} \begin{bmatrix} 1 \\ 1 \end{bmatrix} = \begin{bmatrix} 1 \\ 2 \end{bmatrix} \]
    \pause
\[ \begin{bmatrix} 1 & 0 \\ 0 & 2 \end{bmatrix} \begin{bmatrix} 3 \\ 1 \end{bmatrix} =  \begin{bmatrix} 3 \\ 2 \end{bmatrix} \]
    \pause
 \[ \begin{bmatrix} 1 & 0 \\ 0 & 2 \end{bmatrix} \begin{bmatrix} 2 \\ 3 \end{bmatrix} =  \begin{bmatrix} 2 \\ 6 \end{bmatrix} \]

\column{0.5\tw}

\onslide<2->
\includegraphics[width=0.95\tw]{../Images/Chap2/fig-triangle-start.png}

\end{columns}

\end{frame}

\begin{frame}{Rotation}
    \small

\begin{columns}[T]

\column{0.6\tw}

If we would like to rotate the triangle counter-clockwise by an angle of $\theta$, we have in general
\[ R: \mathbb{R}^2 \to \mathbb{R}^2: \mathbf{x} \mapsto \begin{bmatrix} \cos{\theta} & -\sin{\theta} \\ \sin{\theta} & \cos{\theta} \end{bmatrix} \mathbf{x} .\]
    \pause

For example a counterclockwise rotation by $\theta = \frac{3 \pi}{2}$ would be given by
\[ R: \mathbb{R}^2 \to \mathbb{R}^2: \mathbf{x} \mapsto \begin{bmatrix} \cos{\frac{3 \pi}{2}} & -\sin{\frac{3 \pi}{2}} \\ \sin{\frac{3 \pi}{2}} & \cos{\frac{3 \pi}{2}} \end{bmatrix} \mathbf{x} = \begin{bmatrix} 0 & 1 \\ -1 & 0 \end{bmatrix} \mathbf{x}.\]
    \pause

\column{0.4\tw}

\includegraphics[width=0.8\tw]{../Images/Chap2/fig-triangle-rotate.png}

\end{columns}

    \pause
We have:

\[ \begin{bmatrix} 0 & 1 \\ -1 & 0 \end{bmatrix} \begin{bmatrix} 1 \\ 1 \end{bmatrix} = \begin{bmatrix} 1 \\ -1 \end{bmatrix} , \ \  \pause \begin{bmatrix} 0 & 1 \\ -1 & 0 \end{bmatrix} \begin{bmatrix} 3 \\ 1 \end{bmatrix} =  \begin{bmatrix} 1 \\ -3 \end{bmatrix} , \ \ \pause  \begin{bmatrix} 0 & 1 \\ -1 & 0 \end{bmatrix} \begin{bmatrix} 2 \\ 3 \end{bmatrix} =  \begin{bmatrix} 3 \\ -2 \end{bmatrix} \]


\end{frame}

\begin{frame}{Composing Maps}

Imagine we want to apply three transformations to the triangle in the following order:
\bb
\pause\ii Scale the triangle by a factor of three.
\pause\ii Then vertically shear by a factor of 2 (in vertical direction only).
\pause\ii Finally rotate counter-clockwise by $\theta = 270^{\circ}=\frac{3 \pi}{2}$ radians.
\ee
\pause
\begin{columns}[T]

\column{0.25\tw}

First apply the scaling matrix using matrix $S$:

\[  \begin{bmatrix} 3 & 0 \\ 0 & 3 \end{bmatrix} \begin{bmatrix}1 \\ 1 \end{bmatrix} = \colorb{\begin{bmatrix} 3 \\ 3 \end{bmatrix}} .\]
\pause

\column{0.25\tw}

Then apply the vertical shearing using $V$:

\[ \begin{bmatrix} 1 & 0 \\ 0 & 2 \end{bmatrix} \colorb{ \begin{bmatrix} 3 \\ 3 \end{bmatrix}} = \alert{\begin{bmatrix} 3 \\ 6 \end{bmatrix}}.\]
\pause

\column{0.5\tw}

Finally, we rotate this result counter-clockwise by $\theta = \frac{3 \pi}{2}$ using $R$:

\[ \begin{bmatrix} \cos{\left( \frac{3 \pi}{2} \right)} & -\sin{\left( \frac{3 \pi}{2} \right)} \\ \sin{\left( \frac{3 \pi}{2} \right)} & \cos{\left( \frac{3 \pi}{2} \right)} \end{bmatrix} \alert{ \begin{bmatrix} 3 \\ 6 \end{bmatrix} } = \colorg{ \begin{bmatrix} 6 \\ -3 \end{bmatrix} }.\]

\end{columns}
\pause

\bbox
{\small
The image under the composition of these three transformations is  $R \bigg( V \big( S \mathbf{x} \big) \bigg) = (RVS) \mathbf{x}$}
\ebox

\end{frame}

\begin{frame}{Composing Maps}

Imagine we want to apply three transformations to the triangle in the following order:
\bb
\ii Scale the triangle by a factor of three.
\ii Then vertically shear by a factor of 2 (in vertical direction only).
\ii Finally rotate counter-clockwise by $\theta = 270^{\circ}=\frac{3 \pi}{2}$ radians.
\ee

\begin{columns}[T]

\column{0.6\tw}

\[ M: \mathbb{R}^2 \to \mathbb{R}^2: \mathbf{x} \mapsto RVS \mathbf{x} .\]
\pause
where
\[ RVS =  \begin{bmatrix} \cos{\left( \frac{3 \pi}{2} \right)} & -\sin{\left( \frac{3 \pi}{2} \right)} \\ \sin{\left( \frac{3 \pi}{2} \right)} & \cos{\left( \frac{3 \pi}{2} \right)} \end{bmatrix} \begin{bmatrix} 1 & 0 \\ 0 & 2 \end{bmatrix} \begin{bmatrix} 3 & 0 \\ 0 & 3 \end{bmatrix}  \pause = 
 \begin{bmatrix} 0 & 6 \\ -3 & 0 \end{bmatrix}.\]
Thus we have
\pause
\[ \begin{bmatrix} 0 & 6 \\ -3 & 0 \end{bmatrix} \begin{bmatrix} 1 \\ 1 \end{bmatrix} = \begin{bmatrix} 6 \\ -3 \end{bmatrix} \]

\column{0.4\tw}

\includegraphics[width=0.9\tw]{../Images/Chap2/fig-triangle-combo1.png}

\end{columns}

\end{frame}

\begin{frame}{Composing Linear Transformations}

\bbox
The composite of linear  transformations $A$, $B$, and $C$in the following order:
\bb
\ii First apply transformation given be matrix $A$,
\ii Then apply transformation given be matrix $B$, and 
\ii Finally apply transformation given be matrix $C$
\ee

Is equivalent to the linear transformation given by the product $CBA$. 
\ebox

\pause
\colorb{Note we can extend this idea to compose any number of linear transformations. }

\end{frame}

\begin{frame}{Translations}

To translate the vertex of the triangle at $(1,1)$  to the left by 2 units and up by 3 units, \alert{we add vectors}:

\[ \begin{bmatrix} 1 \\ 1\end{bmatrix} + \begin{bmatrix} -2 \\ 3 \end{bmatrix} = \begin{bmatrix} -1 \\ 4 \end{bmatrix}\]
    \pause

\begin{columns}

\column{0.5\tw}

Doing the same for the other two vertices, we have

\[ \begin{bmatrix} 3 \\ 1\end{bmatrix} + \begin{bmatrix} -2 \\ 3 \end{bmatrix} = \begin{bmatrix} 1 \\ 4 \end{bmatrix} \qquad\pause
\begin{bmatrix} 2 \\ 3\end{bmatrix} + \begin{bmatrix} -2 \\ 3 \end{bmatrix} = \begin{bmatrix} 0 \\ 6 \end{bmatrix}.\]


\column{0.5\tw}

\onslide<2->
\includegraphics[width=0.9\tw]{../Images/Chap2/fig-triangle-shift.png}

\end{columns}

\end{frame}

\begin{frame}{Homogeneous Coordinates}

\bi
\pause\ii Operations such as shearing, scaling, rotations are \alert{linear transformations}. 
\bi
\pause\ii [$\circ$] These operations can be defined by \alert{matrix multiplication}.
\ei
\pause\ii \colorb{A translation is not a linear transformation}. We \colorb{add vectors rather than multiply}.
\pause\ii How can we compose translations with the other linear transformations if translation cannot be represented as matrix multiplication?
\ei

\pause
\bbox
One common way is to introduce \alert{homogeneous coordinates}:

\bi 
\pause\ii Each point $(x,y)$ in $\mathbb{R}^2$ can be identified with the point $(x,y,1)$ in $\mathbb{R}^3$.
\pause\ii We say $(x,y)$ has homogeneous coordinate $(x,y,1)$.
\ei
\ebox

\end{frame}

\begin{frame}{Homogeneous Coordinates}

Now we can define translation by matrix multiplication. For example, if we want to shift the point $(3,1)$ to the left by 2 units and up by the 3 units, then we have the product

\[ \begin{bmatrix}1 & 0 & -2\\ 0 & 1 & 3 \\ 0 & 0 & 1\end{bmatrix} \begin{bmatrix}3 \\ 1\\ 1 \end{bmatrix} = \begin{bmatrix} 3-2 \\ 1+3 \\ 1 \end{bmatrix} = \begin{bmatrix}1\\4\\1 \end{bmatrix}\]
    \pause
\bbox
In general, if we want to translate by $h$ units in the horizontal direction and $k$ units in the vertical direction, using homogeneous coordinates we have:
\pause
\[ \begin{bmatrix} 1 & 0 & h \\ 0 & 1 & k \\ 0 & 0 & 1 \end{bmatrix} \begin{bmatrix}x \\ y \\ 1 \end{bmatrix} = \begin{bmatrix} x+h \\ y+k \\ 1 \end{bmatrix}.\]
\ebox

\end{frame}

\begin{frame}

Let's perform a composition of three different transformations to the triangle in the following order:
\pause

\bb
\pause\ii \alert{Scale the triangle by a factor of 3.}
\pause\ii \colorb{Rotate the triangle counter-clockwise by $\frac{\pi}{2}$.}
\pause\ii \colorg{Translate to the right by 5 units and down by 2 units.}
\ee

We have the following three matrices for the scaling, $S$, rotation, $R$, and translation $B$, respectively,
\pause
\[ \alert{S = \begin{bmatrix} 3 & 0 & 0 \\ 0 & 3 & 0 \\ 0 & 0 & 1 \end{bmatrix}} \qquad\pause
\colorb{R = \begin{bmatrix} \cos{\frac{\pi}{2}} & -\sin{\frac{\pi}{2}}& 0 \\ \sin{\frac{\pi}{2}} & \cos{\frac{\pi}{2}} & 0 \\ 0 & 0 & 1 \end{bmatrix} = \begin{bmatrix} 0 & -1 & 0 \\ 1 & 0 & 0 \\ 0 & 0 & 1 \end{bmatrix}} \qquad\pause
\colorg{B = \begin{bmatrix} 1 & 0 & 5 \\ 0 & 1 & -2 \\ 0 & 0 & 1 \end{bmatrix}}.\]
\pause
Multiplying matrices we can now compose all three maps:

\[\colorg{B}\colorb{R}\alert{S} =  \colorg{\begin{bmatrix} 1 & 0 & 5 \\ 0 & 1 & -2 \\ 0 & 0 & 1 \end{bmatrix}}
\colorb{\begin{bmatrix} 0 & -1 & 0 \\ 1 & 0 & 0 \\ 0 & 0 & 1 \end{bmatrix}}
\begin{bmatrix} 3 & 0 & 0 \\ 0 & 3 & 0 \\ 0 & 0 & 1 \end{bmatrix} = \begin{bmatrix} 0 & -3 & 5\\
3 & 0 & -2 \\ 0 & 0 & 1 \end{bmatrix}.\]
\end{frame}

\begin{frame}

\bb
\ii \alert{Scale the triangle by a factor of 3.}
\pause\ii \colorb{Rotate the triangle counter-clockwise by $\frac{\pi}{2}$.}
\pause\ii \colorg{Translate to the right by 5 units and down by 2 units.}
\ee

\pause
\begin{columns}[T]

\column{0.5\tw}
Then we have each vertex (given in homogeneous coordinates) mapped as follows

$$\begin{bmatrix} 0 & -3 & 5\\
3 & 0 & 2 \\ 0 & 0 & 1 \end{bmatrix} \begin{bmatrix} 1 \\ 1 \\ 1 \end{bmatrix} = \begin{bmatrix} 2 \\ 1 \\ 1 \end{bmatrix}$$

$$\begin{bmatrix} 0 & -3 & 5\\
3 & 0 & 2 \\ 0 & 0 & 1 \end{bmatrix} \begin{bmatrix} 3 \\ 1 \\ 1 \end{bmatrix} \mapsto \begin{bmatrix} 2 \\ 7 \\ 1 \end{bmatrix}$$

$$\begin{bmatrix} 0 & -3 & 5\\
3 & 0 & 2 \\ 0 & 0 & 1 \end{bmatrix} \begin{bmatrix} 2 \\ 3 \\ 1 \end{bmatrix} \mapsto \begin{bmatrix} -4 \\ 4 \\ 1 \end{bmatrix}.$$

\column{0.5\tw}

\onslide<1->
\includegraphics[width=0.9\tw]{../Images/Chap2/fig-triangle-combo2.png}

\end{columns}

\end{frame}

\end{document}

