\documentclass[xcoler=dvipsnames, aspectratio=169]{beamer}
\usepackage{3191Style}
\date{Matrix Operations}

\begin{document}
    \begin{frame}{Representing Matrices}

        \begin{itemize}
            \item If $A\in\R^{m\times n}$, then it as $m$ rows and $n$ columns\pause
            \item The $j^\text{th}$ column vector is denoted $\vec{a}_j$\pause
            \item There are $n$ column vectors, where each $\vec{a}_j$ is in $\R^m$\pause
            \item The entry in the \rTextWait{$i^\text{th}$ row}{5} and $\bTextWait{j^\text{th}}{6}$ column, is 
                called $a_{ij}$.\pause\pause\pause
            \item We \textbf{always} list the row index first then the column\pause
            \item \rText{Note that in Python indexing starts at $0$ while we use $1$ here}
        \end{itemize}
        \onslide<*>
        \[
            A = \begin{bmatrix}\vec{a}_1&\vec{a}_2&\dots&\rTextWait{\vec{a}_j}{2-4}&\dots&\vec{a}_n\end{bmatrix}
                = \begin{bmatrix}
                    a_{11} & a_{12} & \dots & \bTextWait{a_{1j}}{6} & \dots & a_{1n}\\
                    a_{21} & a_{22} & \dots & \bTextWait{a_{2j}}{6} & \dots & a_{2n}\\
                    \vdots & \vdots & \ddots& \bTextWait{\vdots}{6} &       & \vdots\\
                    \rTextWait{a_{i1}}{5} & \rTextWait{a_{i2}}{5} & \dots & \rTextWait{\bTextWait{a_{ij}}{6}}{5} & \dots & \rTextWait{a_{in}}{5}\\
                    \vdots & \vdots &       & \bTextWait{\vdots}{6} & \ddots& \vdots\\
                    a_{m1} & a_{m2} & \dots & \bTextWait{a_{mj}}{6} & \dots & a_{mn}\\
                \end{bmatrix}
        \]
    
    \end{frame}
    \begin{frame}{Special Kinds of Matrices}
        \small
        \begin{itemize}
            \item \rTextWait{Zero Matrix}{1}: A matrix with entries all equal to $0$. Sometimes denoted
                $0_{m\times n}$\pause
            \item \rTextWait{Square Matrix}{2}: A matrix with the same number of rows and columns. (m=n)\pause
            \item \rTextWait{Diagonal Elements}{3}: Elements with the same row and column index. ($a_{ii}$)\pause
            \item \rTextWait{Diagonal Matrix}{4}: A matrix with all elements NOT on the diagonal equal to $0$\pause
            \item \rTextWait{Identity Matrix}{5}: A diagonal matrix with ones on the diagonal. If there are $n$ rows and columns, then we use $I_n$\pause
            \item \rTextWait{Symmetric Matrix}{6}: A matrix satisfying $a_{ij} = a_{ji}$ for all valid $i,j$.\pause\ (\rText{So, it must also be square!})
        \end{itemize}
        \[
            \onslide<1->\rTextWait{0_{2\times 3}=\begin{bmatrix}0&0&0\\0&0&0\end{bmatrix}}{1}\qquad
            \onslide<2->\rTextWait{A=\begin{bmatrix}
            \rTextWait{0}{3}&0&1\\
            2&\rTextWait{5}{3}&1\\
            0&1&\rTextWait{4}{3}
            \end{bmatrix}}{2} \qquad
            \onslide<4->\rTextWait{D=\begin{bmatrix}
                3&0&0\\
                0&1&0\\
                0&0&2
            \end{bmatrix}}{4}\qquad
            \onslide<5->\rTextWait{I_3=\begin{bmatrix}
                1&0&0\\
                0&1&0\\
                0&0&1
            \end{bmatrix}}{5}\qquad
            \onslide<6->\rTextWait{S=\begin{bmatrix}
                1&2&3\\
                2&4&8\\
                3&8&7
            \end{bmatrix}}{6}
        \]
    \end{frame}
    \begin{frame}{Matrix Arithmetic}
        \begin{tcolorbox}
        \begin{itemize}
            \item Let $A,B\in\R^{m\times n}$ (same size and shape!). The \bTextWait{sum}{1}, 
                $C = A+B$ is defined as
                \vspace{-5pt}
                \[
                    c_{ij} = a_{ij} + b_{ij}
                \]
                \vspace{-20pt}
                \pause
            \item Let $A\in\R^{m\times n}$ and $r$ be a scalar. The \bTextWait{scalar multiple}{2}, 
                $C = rA$ is defined as 
                \vspace{-5pt}
                \[
                    c_{ij} = r\cdot a_{ij}
                \]
                \vspace{-20pt}
                \pause
            \item Two matrices, $A\in\R^{m_1\times n_1}$ and $B\in\R^{m_2\times n_2}$ are
                \bText{equal} if
                \begin{enumerate}
                    \item $m_1=m_2$\pause
                    \item $n_1=n_2$ and\pause
                    \item $a_{ij} = b_{ij}$ for all $i,j$.\pause
                \end{enumerate}
        \end{itemize}
        \end{tcolorbox}
        \only<1-5>{\vspace{50pt}}
        \only<6>{
        \vspace{-10pt}
        \begin{example}
            Let $A = \begin{bmatrix}-1&2&1\\0&2&4\end{bmatrix}, 
                B= \begin{bmatrix}1 & 3 & 4\\0&4&2\end{bmatrix}$. Compute $2A-B$.
        \end{example}
    }
    \end{frame}
    \begin{frame}{Matrix Addition \& Scalar Multiplication Properties}
        \begin{tcolorbox}
            For each of the following properties, let $A,B,C\in\R^{m\times n}$, and $r,s$ be 
            scalars
            \begin{columns}
                \column{.5\textwidth}
                \begin{enumerate}
                    \item $A+B = B+A$ \pause
                    \item $(A+B) + C = A + (B+C)$ \pause
                    \item $A + 0_{m\times n} = A$\pause
                \end{enumerate}
                \column{.5\textwidth}
                \begin{enumerate}
                    \addtocounter{enumi}{3}
                    \item $r(A+B) = rA + rB$ \pause
                    \item $(r+s)A = rA + sA$ \pause
                    \item $(r)sA = (rs)A$
                \end{enumerate}
            \end{columns}
        \end{tcolorbox}
    \end{frame}
    \begin{frame}{The Transpose}
        \begin{defn}
            Let $A\in\R^{m\times n}$. The \bText{transpose} of $A$, denoted $A^\top\in\R^{n\times m}$
            is the matrix with columns formed from rows of $A$. IE:
            \[
                a^\top_{ij} = a_{ji}
            \]
            for all valid $i,j$.
        \end{defn}\pause
        \begin{columns}
            \column{.5\textwidth}
            \begin{example}
                Give the transpose of
                \[
                    A = \begin{bmatrix}
                        1 & 2 & 3\\
                        -4&-5 &-6
                    \end{bmatrix}
                \]
            \end{example}\pause
            \column{.5\textwidth}
                \[
                    A^\top = \begin{bmatrix}
                        1 & -4\\
                        2 & -5\\
                        3 & -6
                    \end{bmatrix}
                \]
        \end{columns}
    \end{frame}
    \begin{frame}{Matrix Multiplication and Linear Transformations}
        \small
        \begin{defn} \rTextWait{Composition of Linear Transformations}{1-4}:
            Let $T:\R^m\rightarrow\R^p$ and $S:\R^n\rightarrow\R^m$ be linear transformations defined
            by 
            \[
                T(\vec{y}) = A\vec{y}\textnormal{ and } S(\vec{x}) = B\vec{x}
            \]\pause\ 
            then we define the \bTextWait{composition}{1-4} of $T$ and $S$ to be
            \[
                (T\circ S)(\vec{x}) = T(S(\vec{x})) \pause\ = A(B\vec{x}) \pause\ = AB\vec{x}
            \]
        \end{defn}\pause\ 
        \begin{tcolorbox}
            Note: $AB$ is only defined with $A$ has the same number of \rText{rows} as $B$ has 
            \bText{columns}
        \end{tcolorbox}
    \end{frame}
    \begin{frame}{Matrix Multiplication}
        \begin{defn}
            \rText{Matrix Product}: Let $A\in\R^{m\times n}$, $B\in\R^{n\times p}$. 
            We define the product of two matrices $C=AB$ to be the matrix such that for all
            $\vec{x}\in\R^{p}$ such that $C\vec{x} = A(B\vec{x})$.
        \end{defn}
        \pause
        There are $3$ main ways to compute the \bText{matrix product}, $AB$.\pause
        \begin{enumerate}
            \item Column-wise\pause
            \item Component-wise\pause
            \item Sums of other matrices
        \end{enumerate}
    \end{frame}
    \begin{frame}{Matrix Multiplication Method 1 (Column-wise)}
        \small
        \begin{tcolorbox}
            Let $A\in\R^{m\times n},B\in\R^{n\times p}$. Then we define the \bText{matrix product}
            as: 
            \[
                AB = A\begin{bmatrix} \vec{b}_1 & \vec{b}_2 & \dots & \vec{b}_p\end{bmatrix} =
                    \begin{bmatrix} A\vec{b}_1 & A\vec{b}_2 & \dots & A\vec{b}_p\end{bmatrix}
            \]
            where $\vec{b}_k\in\R^n$ are the \rText{columns} of $B$.
        \end{tcolorbox}\pause
        \vspace{-10pt}
        \begin{example}
            Let $A = \begin{bmatrix}1&-4\\2&-5\\3&-6\end{bmatrix}$ and $B = \begin{bmatrix}
                2&1\\1&4
        \end{bmatrix}$. Then \pause
        \[
            C = AB = \begin{bmatrix}\begin{bmatrix}1&-4\\2&-5\\3&-6\end{bmatrix}\begin{bmatrix}
    2\\1\end{bmatrix}
                & \begin{bmatrix}1&-4\\2&-5\\3&-6\end{bmatrix}\begin{bmatrix}1\\4\end{bmatrix}\end{bmatrix}\pause
                    = \begin{bmatrix}
                        -2 & \only<5> {-15}\only<4>{\quad\,\,}\\
                        -1 & \only<5> {-18}\\
                         0 & \only<5> {-21}
                    \end{bmatrix}
        \]
        \end{example}
    \end{frame}
    \begin{frame}{Matrix Multiplication Method 2 (Component-wise)}
        \begin{tcolorbox}
            Let $A\in\R^{m\times n},B\in\R^{n\times p}$. Then we define the \bText{matrix product}
            as the matrix $C=AB$ where for all $i,j$: 
            \[
                c_{ij} = a_{i1}b_{1j} + a_{i2}b_{2j} + \cdots + a_{in}b_{nj}
            \]
        \end{tcolorbox}\pause
        \vspace{-10pt}
        \begin{example}
            Let $A = \begin{bmatrix}\rTextWait{1}{4}&\rTextWait{-4}{4}\\\
            2&-5\\3&-6\end{bmatrix}$ and $B = \begin{bmatrix}
                \bTextWait{2}{4}&1\\\bTextWait{1}{4}&4
        \end{bmatrix}$. Then \pause
        \[
            C=AB= \begin{bmatrix}
                c_{11} & c_{12}\\
                c_{21} & c_{22}\\
                c_{31} & c_{32}
            \end{bmatrix}\pause =
            \begin{bmatrix}
                \rTextWait{1}{4}\cdot \bTextWait{2}{4}+ \rTextWait{-4}{4}\cdot \bTextWait{1}{4} & 1\cdot1 + -4\cdot4\\
                2\cdot 2+ -5\cdot 1 & 2\cdot1 + -5\cdot4\\
                3\cdot 2+ -6\cdot 1 & 3\cdot1 + -6\cdot4
            \end{bmatrix}\pause = \begin{bmatrix}
                -2 & -15\\
                -1 & -18\\
                0  & -21
            \end{bmatrix}
        \]
        \end{example}
    \end{frame}
    \begin{frame}{Matrix Multiplication Method 3 (Sums of other matrices)}
        \small
        \begin{tcolorbox}
            Let $A\in\R^{m\times n},B\in\R^{n\times p}$. Then we define the \bText{matrix product}
            as: 
            \[
                AB = \begin{bmatrix} \vec{a}_1 & \dots & \vec{a}_n\end{bmatrix} \begin{bmatrix}
                    \vec{b}_1^\top\\\vdots\\\vec{b}_n^\top
            \end{bmatrix} = \vec{a}_1\vec{b}_1^\top + \cdots + \vec{a}_n\vec{b}_n^\top
            \]
            where $\vec{a}_i\in\R^{m\times 1}$ are the \rText{columns} of $A$ and 
            $\vec{b}_j^\top\in\R^{1\times p}$ are the \bText{rows} of $B$ (\rText{columns} of $B^\top$!)
        \end{tcolorbox}\pause
    \end{frame}
    \begin{frame}{Method 3 Example}
        \begin{example}
            Let $A = \begin{bmatrix}
                \rTextWait{1}{2}&\rTextWait{-4}{3}\\
                \rTextWait{2}{2}&\rTextWait{-5}{3}\\
                \rTextWait{3}{2}&\rTextWait{-6}{3}\end{bmatrix}$ and $B = \begin{bmatrix}
                    \bTextWait{2}{2}&\bTextWait{1}{2}\\\bTextWait{1}{3}&\bTextWait{4}{3}
        \end{bmatrix}$. Then \pause
        \[
            C = AB = \begin{bmatrix}
                \rTextWait{1}{2}\\
                \rTextWait{2}{2}\\
                \rTextWait{3}{2}\end{bmatrix}
                \begin{bmatrix}\bTextWait{2}{2}&\bTextWait{1}{2}\end{bmatrix} + 
                \begin{bmatrix}
                \rTextWait{-4}{3}\\
                \rTextWait{-5}{3}\\
                \rTextWait{-6}{3}\end{bmatrix}
                \begin{bmatrix}\bTextWait{1}{3}&\bTextWait{4}{3}\end{bmatrix}\pause\pause = 
                    \begin{bmatrix}
                        2 & 1\\
                        4 & 2\\
                        6 & 3
                    \end{bmatrix} + \begin{bmatrix}
                        -4&-16\\
                        -5&-20\\
                        -6&-24\\
                    \end{bmatrix}\pause = \begin{bmatrix}
                        -2 & -15\\
                        -1 & -18\\
                        0  & -21
                    \end{bmatrix}
        \]
        \end{example}
    \end{frame}
    \begin{frame}{Is It The Correct Shape?}

        {\scriptsize
        \[
            A=\begin{bmatrix}3&0\\-1&2\\1&1\end{bmatrix}, B=\begin{bmatrix}4&-1\\0&2\end{bmatrix},
            C=\begin{bmatrix}1&4&2\\3&1&5\end{bmatrix}, D = \begin{bmatrix}1&0&0\\0&1&0\\0&0&1\end{bmatrix},
            E=\begin{bmatrix}4&6\\0&3\\1&0\end{bmatrix}, F=\begin{bmatrix}1&4\\2&3\end{bmatrix}
        \]}
        \begin{columns}
            \column{.33\textwidth}
                \includegraphics[scale=.125]{images/manhole.png}
                \pause
                \vspace{100pt}
            \column{.67\textwidth}
            \vspace{-50pt}
            \begin{enumerate}
                \item For which matrices is \rText{addition} with $A$ defined?\pause
                \item For which matrices is \rText{addition} with $B$ defined?\pause
                \item For which matrices is \rText{addition} with $C$ defined?\pause
                \item For which matrices is \bText{multiplication} with $A$ defined?\pause
                \item For which matrices is \bText{multiplication} with $B$ defined?\pause
                \item For which matrices is \bText{multiplication} with $C$ defined?
            \end{enumerate}
        \end{columns}
    \end{frame}
    \begin{frame}{Example}

        \scriptsize
        \[
            A=\begin{bmatrix}3&0\\-1&2\\1&1\end{bmatrix}, B=\begin{bmatrix}4&-1\\0&2\end{bmatrix},
            C=\begin{bmatrix}1&4&2\\3&1&5\end{bmatrix}, D = \begin{bmatrix}1&0&0\\0&1&0\\0&0&1\end{bmatrix},
            E=\begin{bmatrix}4&6\\0&3\\1&0\end{bmatrix}, F=\begin{bmatrix}1&4\\2&3\end{bmatrix}
        \]
        Compute (if possible):
        \begin{columns}
            \column{.5\textwidth}
            \only<1->{
                \begin{enumerate}
                    \item $AF$
                \end{enumerate}
            }
            \only<2->{
                This is defined!
            }
            \only<4->{
                \[
                    AF = \begin{bmatrix}
                        3&0\\
                        -1&2\\
                        1&1
                    \end{bmatrix}\begin{bmatrix}
                        1&4\\
                        2&3
                    \end{bmatrix} \only<5->{= \begin{bmatrix}
                        3\\-1\\1
                    \end{bmatrix}\begin{bmatrix}1&4\end{bmatrix} + \begin{bmatrix}
                        0\\2\\1
                    \end{bmatrix}\begin{bmatrix}2&3\end{bmatrix}}
                \]
                \only<4,5>{
                    \vspace{20pt}
                }
            }
            \only<6->{
                \[
                    = \begin{bmatrix}
                        3 & 12\\
                        -1&-4\\
                        1 & 4
                    \end{bmatrix} + \begin{bmatrix}
                        0 & 0\\
                        4 & 6\\
                        2 & 3
                    \end{bmatrix}
                    \only<7->{
                        = \begin{bmatrix}
                            3 & 12\\
                            3 & 2 \\
                            3 & 7
                        \end{bmatrix}
                    }
                \]
            }
            \column{.5\textwidth}
            \only<1->{
                \begin{enumerate}
                    \addtocounter{enumi}{1}
                    \item $FA$
                \end{enumerate}
                \only<3->{
                    Not defined!
                }
                \only<4->{
                    \vspace{80pt}
                }
            }
        \end{columns}
        \vspace{120pt}
    \end{frame}
    \begin{frame}{Now You Try!}

        {\scriptsize
        \[
            A=\begin{bmatrix}3&0\\-1&2\\1&1\end{bmatrix}, B=\begin{bmatrix}4&-1\\0&2\end{bmatrix},
            C=\begin{bmatrix}1&4&2\\3&1&5\end{bmatrix}, D = \begin{bmatrix}1&0&0\\0&1&0\\0&0&1\end{bmatrix},
            E=\begin{bmatrix}4&6\\0&3\\1&0\end{bmatrix}, F=\begin{bmatrix}1&4\\2&3\end{bmatrix}
        \]}
        \begin{columns}
            \column{.5\textwidth}
            \begin{enumerate}
                \item $BF$
                    \iftoggle{showSolutions}{
                        \only<2->{
                            \[
                                BF = \begin{bmatrix}
                                    2&13\\
                                    4&6
                                \end{bmatrix}
                            \]
                        }
                    }{}
            \end{enumerate}
            \column{.5\textwidth}
            \begin{enumerate}
                \addtocounter{enumi}{1}
                \item $FB$
                    \iftoggle{showSolutions}{
                        \only<3->{
                            \[
                                FB = \begin{bmatrix}
                                    4&7\\
                                    8&4
                                \end{bmatrix}
                            \]
                        }
                    }{}
            \end{enumerate}
        \end{columns}
        \vspace{120pt}
    \end{frame}
    \begin{frame}{Multiplication and Transpose Properties}
        \begin{tcolorbox}
            Let $A,B\in\R^{m\times n}$, $C\in\R^{n\times p}$, and $r$ be a scalar.
            \begin{enumerate}
                \item $\left(A^\top\right)^\top = A$\pause
                \item $(A+B)^\top = A^\top + B^\top$\pause
                \item $(rA)^\top = rA^\top$\pause
                \item $(AB)^\top = B^\top A^\top$
            \end{enumerate}
        \end{tcolorbox}
    \end{frame}
\end{document}
