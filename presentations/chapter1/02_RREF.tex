\documentclass[xcoler=dvipsnames, aspectratio=169]{beamer}
\usepackage{3191Style}
\date{Row Echelon Form}
\newcommand{\INPUT}{\textbf{Input:}}
\newcommand{\OUTPUT}{\textbf{Output:}}

\begin{document}
    \begin{frame}{A General Algorithm}
        Let's build a more systematic way of solving linear systems (that we can hopefully 
        have a computer do for us!).
        We want an algorithm that we can:
        \pause
        \begin{enumerate}
            \item Apply to any augmented matrix from a general system of equations
                \pause
            \item Transform to a form where we can find the solution(s) (if any exist!) by quickly looking
                \pause
            \item Using only Elementary Row Operations
        \end{enumerate}
        \pause

        \[
            \aMat{ccc|c}{
                1 & 0 & 0 & 1\\
                0 & 1 & 0 & 2\\
                0 & 0 & 1 & 4
            }\qquad\qquad
            \aMat{cc|c}{
                1 & 0 & -2\\
                0 & 1 & 8\\
            }
        \]

        \begin{tcolorbox}
            What are the similarities between these two augmented matrices?
        \end{tcolorbox}
    \end{frame}

    \begin{frame}{Row Echelon Form}
        \small
        \begin{defn}
            Row Echelon Form. We say that a matrix is in Row Echelon Form if and only if
            it satisfies the following requirements
            \begin{enumerate}
                \item All non-zero rows are above any rows of all zeros.
                \item Each leading entry of a row is in a column to the right of the leading
                    entry of the row above it.
                \item All entries in a column below a leading entry are zeros.
            \end{enumerate}
        \end{defn}
        \pause
            \begin{example}
            \begin{center}
                \includegraphics[width=4in]{../Images/Chap1/fig-row-echelon.png}
            \end{center}
            \end{example}
    \end{frame}
    \begin{frame}{Reduced Row Echelon Form}
        \small
        \begin{defn}
            Reduced Row Echelon Form (RREF). We say that a matrix is in RREF if and only if
            it satisfies the requirements for Row Echelon Form and 
            \begin{enumerate}
                \addtocounter{enumi}{3}
                \item The leading entry in each non-zero row is $1$.
                \item Each leading $1$ is the only non-zero in its column.
            \end{enumerate}
        \end{defn}
        \pause
        \begin{example}
            \begin{center}
                \includegraphics[width=2.5in]{../Images/Chap1/fig-row-echelon2.png}\qquad\quad
                \includegraphics[width=2.5in]{../Images/Chap1/fig-reduced-row2.png}
            \end{center}
        \end{example}
        \pause
        What kind of transformations can take us from the left to the right?
    \end{frame}

    \begin{frame}{Formalizing Our Algorithm}
        We've been doing this already actually! In order to better formalize it, let's break down
        our algorithm into $2$ smaller steps.

        \begin{algorithm}[H]
            \small
            \caption{RREF Transformation}
            \INPUT A matrix $A$ with $m$ rows and $n$ columns of real numbers\\
            \OUTPUT $A$ transformed to RREF using Only Elementary Row Operations
            \begin{algorithmic}[1]
                \STATE Reduce $A$ to Row Echelon Form
                \STATE Reduce $A$ to RREF
            \end{algorithmic}
        \end{algorithm}
    \end{frame}
    \begin{frame}{Reducing to Row Echelon Form}
        Let's start with an example. Lets reduce the following augmented matrix to Row Echelon Form.
        \begin{columns}
            \column{.5\textwidth}
            \begin{enumerate}
                \onslide<1->
                \item Start with the leftmost non-zero column. We call this the \rText{pivot column}.
                \onslide<2->
                \item Pick some non-zero element. This is
                    called the \rText{pivot}. Move it to the top if necessary.
                \onslide<3->
                \item Create zeros in all positions below the pivot.
            \end{enumerate}
            \column{.5\textwidth}
            \onslide<1->
        \[
                \aMat{ccc|c}{
                    0 & 1 & 2 & 3 \\
                    2 & 8 & 1 & 2 \\
                    1 & 4 & 1 & 1
                }
            \]
            \onslide<2->
            \[
                \aMat{ccc|c}{
                    \bTextWait{1}{2} & \bTextWait{4}{2} & \bTextWait{1}{2} & \bTextWait{1}{2} \\
                    2 & 8 & 1 & 2 \\
                    \bTextWait{0}{2} & \bTextWait{1}{2} & \bTextWait{2}{2} & \bTextWait{3}{2}
                }
            \]
            \onslide<3->
            \[
                \aMat{ccc|c}{
                    1 & 4 & 1 & 1 \\
                    \bText{0} & \bText{0} & \bText{-1} & \bText{0} \\
                    0 & 1 & 2 & 3
                }
            \]
        \end{columns}
    \end{frame}
    \begin{frame}{Reducing to Row Echelon Form}
        \begin{columns}
            \column{.5\textwidth}
            \begin{enumerate}
                \addtocounter{enumi}{3}
            \item Ignore the row containing the pivot and consider the remaining submatrix (highlighted in red). Then repeat the steps from before again!
            \end{enumerate}
            \column{.5\textwidth}
            \[
                \aMat{ccc|c}{
                    1 & 4 & 1 & 1 \\
                    0 & \rText{0} & \rText{-1} & \rText{0} \\
                    0 & \rText{1} & \rText{2} & \rText{3}
                }
            \]
        \end{columns}
        \pause
        \[
            \aMat{ccc|c}{
                1 & 4 & 1 & 1 \\
                0 & 0 & -1 & 0 \\
                0 & 1 & 2 & 3
            } \rightarrow \aMat{ccc|c}{
                1 & 4 & 1 & 1 \\
                0 & 1 & 2 & 3 \\
                0 & 0 & -1 & 0 
            }
        \]
    \end{frame}
    \begin{frame}{Reducing to RREF}
        Now that we are in Row Echelon, we're almost there! We have $2$ more conditions to meet.
        \pause
        \begin{enumerate}
            \item All our pivots are $1$
            \item All elements above our pivots are $0$.
        \end{enumerate}
        \pause
        So, let's fix them!
    \end{frame}
    \begin{frame}{Reducing to RREF}
        \small
        \begin{columns}
            \column{.5\textwidth}
            \begin{enumerate}
                \addtocounter{enumi}{4}
                \onslide<1->
                \item Scale each row to make all pivots equal to $1$.
                \onslide<2->
                \item Starting with the rightmost pivot column, zero out all elements above each non-zero pivot using Elementary Row Operations.
            \end{enumerate}
            \column{.5\textwidth}
            \onslide<1->
            \[
                \aMat{ccc|c}{
                    1 & 4 & 1 & 1 \\
                    0 & 1 & 2 & 3 \\
                    \bText{0} & \bText{0} & \bText{1} & \bText{0}
            }
            \]
            \onslide<2->
            \[
                \aMat{ccc|c}{
                    1 & 4 & \rText{0} & \rText{1} \\
                    0 & 1 & \rText{0} & \rText{3} \\
                    0 & 0 & 1 & 0 
            }\rightarrow
                \aMat{ccc|c}{
                    1 & \rText{0} & 0 & \rText{-11} \\
                    0 & 1 & 0 & 3 \\
                    0 & 0 & 1 & 0 
            }
            \]
        \end{columns}
        Then our answer is \[(x_1,x_2,x_3)=(-11,3,0)\].
    \end{frame}
    \begin{frame}{Gaussian Elimination\footnote{\url{https://en.wikipedia.org/wiki/Gaussian\_elimination\#History}}}
        \begin{algorithm}[H]
            \small
            \caption{Gaussian Elimination}
            \INPUT A matrix $A$ with $m$ rows and $n$ columns of real numbers\\
            \OUTPUT $A$ transformed to RREF
            \begin{algorithmic}[1]
                \STATE Consider the leftmost column. Called the \rText{pivot column}
                \STATE Pick a non-zero entry as the \rText{pivot}. Exchange rows if needed.
                \STATE Use Elementary Row Operations to create zeros below the pivot
                \STATE Ignore the row containing the pivot. Repeat the $3$ steps above on the remaining submatrix until there are no more non-zero rows.
                \STATE Scale each row so each non-zero pivot is $1$.
                \STATE Cancel out all elements above each pivot with Elementary Row Operations in each pivot column.
            \end{algorithmic}
        \end{algorithm}
    \end{frame}
    \begin{frame}{Existence and Uniqueness of RREF}
        \small
        \begin{enumerate}
            \item \rText{Existence}: Given any matrix, we can apply a series of elementary row operations to find an equivalent RREF.
                \pause
            \item The series of operations are not unique, but the result is unique!
                \pause
                \[
                    \footnotesize
                    \aMat{cc|c}{
                        0 & 1 & -2\\
                        -2&-2 & -4\\
                        -1 &0 & -4
                    }\rightarrow\aMat{cc|c}{
                        -2&-2 & -4\\
                        0 & 1 & -2\\
                        -1 &0 & -4
                    }\rightarrow\aMat{cc|c}{
                        1 & 1 &  2\\
                        0 & 1 & -2\\
                        -1 &0 & -4
                    }\rightarrow\aMat{cc|c}{
                        1 & 1 &  2\\
                        0 & 1 & -2\\
                        0 & 0 &  0
                    }\rightarrow\aMat{cc|c}{
                        1 & 0 &  4\\
                        0 & 1 & -2\\
                        0 & 0 &  0
                    }
                \]
                \pause
                \[
                    \scriptsize
                    \aMat{cc|c}{
                        0 & 1 & -2\\
                        -2&-2 & -4\\
                        -1 &0 & -4
                    }\rightarrow\aMat{cc|c}{
                        -1 &0 & -4\\
                        -2&-2 & -4\\
                        0 & 1 & -2
                    }\rightarrow\aMat{cc|c}{
                        1 &0 & 4\\
                        -2&-2 & -4\\
                        0 & 1 & -2
                    }\rightarrow\aMat{cc|c}{
                        1 &0 & 4\\
                        0 & 1 & -2\\
                        -2&-2 & -4
                    }\rightarrow\aMat{cc|c}{
                        1 &0 & 4\\
                        0 & 1 & -2\\
                        0 & 0 & 0
                    }
                \]
        \end{enumerate}
                \pause
        \begin{theorem}
            Every matrix is equivalent to one and only one reduced RREF matrix.
        \end{theorem}
    \end{frame}
    \begin{frame}{Example}
        \small
        Find a solution to the system
        \begin{alignat*}{4}
            x_1 &+ 2x_2 &+ x_3 &= -2\\
            x_1 &+ 3x_2 &- 2x_3 &= 1
        \end{alignat*}
        Using Gaussian Elimination, we get
        \[
            \aMat{ccc|c}{
                1 & 2 & 1 & -2\\
                1 & 3 & -2& 1
            }\rightarrow\aMat{ccc|c}{
                1 & 2 & 1 & -2\\
                0 & 1 & -3& 3
            }\rightarrow\aMat{ccc|c}{
                \rTextWait{1}{2} & 0 & 7 & -8\\
                0 & \bTextWait{1}{2} & -3& 3
            }
        \]
        \pause
        This gives the solution set:
        \begin{alignat*}{4}
            \rText{x_1} &\,  &+ 7x_3 &= -8\\
            \,  &\, \bText{x_2} &- 3x_3 &= 3
        \end{alignat*} or in parametric form:
        \[
            (-8-7x_3,3+3x_3,x_3) = \begin{cases}
                \rText{x_1} &= -8-7x_3\\
                \bText{x_2} &= 3+3x_3\\
                x_3&\text{ is free}
            \end{cases}
        \]
    \end{frame}
    \begin{frame}{Parametric Form}
        \begin{defn}
            \rText{Parametric Form}: We say that a solution to a system of linear equations is in
            \bText{parametric form} if it has a variable in the ordered pair.
        \end{defn}
        \begin{example}
        For example
            \[
                (-8-7x_3,3+3x_3,x_3)
            \]
            is in parametric form because $x_3$ is still a variable and we don't know what it is!
            We will usually write either the variable with a subscript like above or using another variable.
            Like:
            \[
                (-8-7s,3+3s,s)
            \]
        \end{example}
    \end{frame}
    \begin{frame}{Solutions of Linear Systems}
        We have
        \[
            \begin{cases}
                x_1 &= -8-7x_3\\
                x_2 &= 3+3x_3\\
                x_3&\text{ is free}
            \end{cases}.
        \]
        \begin{itemize}
            \item Variables $x_1$ and $x_2$ came from \rText{pivot columns}, and we call these \bText{basic variables}
            \item Variable $x_3$ did not come from a pivot column, so we call it a \bText{free variable}. This means it take on any value we want (IE we are free to choose it)
            \item If we have at least one free variable, then we have an infinite number of solutions.
        \end{itemize}
    \end{frame}
    \begin{frame}{Another Example}
        \small
        Find a solution to the system
        \begin{alignat*}{4}
            \, &\,& x_2 &+ 2x_3 &= 0\\
            x_1&+& 2x_2 &+ 3x_3&= 1\\
            x_1&+&  x_2 &+ x_3&= 2
        \end{alignat*}
        Gaussian Elimination gives us
        \[
            \aMat{ccc|c}{
                0 & 1 & 2 & 0 \\
                1 & 2 & 3 & 1 \\
                1 & 1 & 1 & 2
            }\rightarrow\aMat{ccc|c}{
                1 & 1 & 1 & 2 \\
                1 & 2 & 3 & 1 \\
                0 & 1 & 2 & 0 
            }\rightarrow\aMat{ccc|c}{
                1 & 1 & 1 & 2 \\
                0 & 1 & 2 & -1 \\
                0 & 1 & 2 & 0 
            }\rightarrow\aMat{ccc|c}{
                \bText{1} & 1 & 1 & 2 \\
                0 & \bText{1} & 2 & -1 \\
                0 & 0 & 0 & \rText{1} 
            }
        \]
        This is an inconsistent system!
        \begin{alignat*}{4}
            \bText{x_1} &\,& \, &-& x_3 &= 0\\
            \,&\,& \bText{x_2} &+& 2x_3&= 0\\
            \,&\,& \, &\,& 0&= 1
        \end{alignat*}
    \end{frame}
    \begin{frame}{Summary}
        \begin{tcolorbox}
            For a given system of linear equations:
            \begin{enumerate}
                \item Write the augmented matrix
                \item Use Elementary Row Operations to reduce to Row Echelon Form
                    \begin{enumerate}
                        \item \rTextWait{If the rightmost column is a pivot column, then it is inconsistent, and no solution exists. We can now stop}{2}
                        \item If the rightmost column is not a pivot column, then it is consistent, and we have a solution, so we need to continue
                    \end{enumerate}
                \item Reduce to RREF
                \item Rewrite as a system of linear equations
                \item Put all basic variables on the left and all free variables on the right.
            \end{enumerate}
        \end{tcolorbox}
    \end{frame}
\end{document}
